\documentclass[12pt]{article}

\title{Linear Algebra Transformations}
\date{}

\usepackage{graphicx} % LaTeX package to import graphics
\graphicspath{{Math/linear_algebra/images}} % Configuring the graphicx package

\usepackage[a4paper, margin=1.5cm, footskip=.5cm]{geometry}
\usepackage{amssymb}
\usepackage{amsmath}
\usepackage{parskip}
\usepackage{tocloft}
\usepackage{float}

\renewcommand{\cftsecleader}{\cftdotfill{\cftdotsep}} % Adds dots to TOC
\setcounter{secnumdepth}{0} % sections are level 1


% Set all sections to have centered headings
\usepackage{titlesec}
\titleformat{\section}  % which section command to format
    {\centering\fontsize{17}{19}\bfseries} % format for whole line
    {\thesection} % how to show number
    {1em} % space between number and text
    {} % formatting for just the text
    [] % formatting for after the text
% --------------------------------------------


% Macro to reduce code repetition
%
% \NewDocumentCommand{\macroname}{argument specifiers}{%
%     expansion code
% }
% 
% "m" is a standard mandatory argument, which can be a single token or multiple tokens surrounded by {}
% "o" is a standard optional argument, it will supply -NoValue- marker if not given
% "O", given as O{(default)}, is like "o" but returns default if no valus is given
% "v" reads an argument ‘verbatim’

\NewDocumentCommand{\Vector}{ m m}{%
    \ensuremath{%
        \begin{bmatrix}
                #1 \\ 
                #2
        \end{bmatrix}
    }
}

\NewDocumentCommand{\VectorThree}{ m m m}{%
    \ensuremath{%
        \begin{bmatrix}
                #1 \\ 
                #2 \\
                #3 
        \end{bmatrix}
    }
}

\NewDocumentCommand{\InlineFigure}{ O{0.70\textwidth} O{0.35\textwidth} m m v }{%
    \begin{figure}[H]
        \centering
        \includegraphics[width=#1, height=#2]{#5}
        \caption{#3}
        \label{#4}
    \end{figure}
}
\begin{document}
    \maketitle

    \tableofcontents

\newpage

\section{Normal Vector}

\InlineFigure{Normal Vector}{fig1}{normal_vector.png}

The normal vector, often simply called the \textit{normal} to a surface is a vector which is perpendicular to the surface at a given point. When normals are considered on closed surfaces, the inward-pointing normal (pointing towards the interior of the surface) and outward-pointing normal are usually distinguished.

The unit vector obtained by normalizing the normal vector (i.e. dividing a nonzero normal vector by its vector norm) is the unit normal vector, often known simply as the \textit{unit normal}. Care should be taken to not confuse the terms \textit{vector norm} (Length of vector), \textit{normal vector} (Perpendicular vector) and \textit{normalized vector} (Unit-length vector).

The normal vector is commonly denoted $N$, with a hat sometimes (but not always) added ($\hat{N}$) to explicitly indicate a unit normal vector.


\section{Normalized Vector}

The normalized vector of $X$ is a vector in the same direction but with norm (length) of 1. It is denoted $\hat{X}$ and given by;

\begin{center}
    $\hat{X} = \frac{X}{\vert X \vert}$
\end{center}

Where $\vert X \vert$ is the norm of $X$. It is also called a \textbf{unit vector}.

\section{Vector Norm}

In mathematics, the \textit{norm} of a vector is its \textbf{length}. A vector is a mathematical object that has a size, called the \textbf{magnitude} and a direction. For the real numbers, the only norm is the absolute value. For spaces with more dimensions, the norm can be any function \textit{p} with the following three properties:

\begin{enumerate}
    \item Scales for real numbers \textit{a}, that is, $p(ax) = \vert a \vert p(x)$
    \item Function of sum is less than sum of functions, that is, $p(x + y) \leq p(x) + p(y)$ (also known as the triangle equality)
    \item $p(x) = 0$ if and only if $x = 0$
\end{enumerate}

For a vector $x$, the associated norm is written as $\| x \|_p$ or $L_p$ where $p$ is some value. The value of the norm of $x$ with some length $N$ is as follows:

\begin{center}
    $\| x \|_p$ = $\sqrt[p]{x^p_1 + x^p_2 + ... + x^p_N}$
\end{center}

The most common usage of this is the Euclidean norm, also called the standard distance formula.

e.g. 

\begin{enumerate}
    \item The one-norm is the sum of absolute values: $\| X \|_1$ = $\vert x_1 \vert$ + $\vert x_2 \vert$ + $...$ + $\vert x_N \vert$. This is like finding the distance from one place on a grid to another by summing together the distances in all directions the grid goes; see Manhatten Distance.
    
    \InlineFigure{Manhatten Distance}{fig2}{manhatten_distance.png}
    
    $L_1$, written as $\| X \|_1$, is defined as $\| X \|_1$ = $\sum\limits_{i=1}^{n} \vert X_i \vert$
    
    \item Euclidean norm (also called L2-norm) is the sum of the squares of the values:
    
    \begin{center}
        $\| x \|_2$ = $\sqrt[p]{x^2_1 + x^2_2 + ... + x^2_N}$
    \end{center}
    
    \InlineFigure{Euclidean Norm}{fig3}{euclidean_norm.png}
    
    A visual difference between the Euclidean Norm and the Manhatten Norm is shown in figure 4.
    
    \InlineFigure{Euclidean vs Manhatten}{fig4}{euclidean_vs_manhatten.png}
    
    \item Maximum norm is the maximum absolute value: $\| X \|_\infty$ = $max(\vert X_1 \vert, \vert X_2 \vert, ... \vert X_N \vert)$
    
    
\end{enumerate}

\end{document}